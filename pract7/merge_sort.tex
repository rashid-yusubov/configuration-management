% ===> this file was generated automatically by noweave --- better not edit it

\documentclass{report}
\usepackage[utf8]{inputenc}
\usepackage[russian]{babel}
\usepackage{listings}  % Подключение пакета для кода

\begin{document}

\chapter{Алгоритм сортировки слиянием}

Этот документ описывает алгоритм сортировки слиянием с использованием noweb.

\section{Описание алгоритма}

Алгоритм сортировки слиянием разделяет список на две половины, рекурсивно сортирует их, а затем сливает отсортированные половины.

\subsection{Шаги алгоритма}

1. Разделить список на две половины.
2. Рекурсивно отсортировать каждую из половин.
3. Слить отсортированные половины в один отсортированный список.

\section{Исходный код}

\subsection{Функция слияния}

\begin{lstlisting}[language=Haskell]
merge :: Ord a => [a] -> [a] -> [a]
merge [] ys = ys
merge xs [] = xs
merge (x:xs) (y:ys)
  | x < y     = x : merge xs (y:ys)
  | otherwise = y : merge (x:xs) ys
\end{lstlisting}

\subsection{Основная функция сортировки слиянием}

\begin{lstlisting}[language=Haskell]
mergeSort :: Ord a => [a] -> [a]
mergeSort [] = []
mergeSort [x] = [x]
mergeSort xs = merge (mergeSort left) (mergeSort right)
  where
    (left, right) = splitAt (length xs `div` 2) xs
\end{lstlisting}

\section{Заключение}

Алгоритм сортировки слиянием эффективно сортирует элементы, рекурсивно деля их и объединяя.

\end{document}

\nwfilename{merge_sort.nw}
